

\documentclass[a4paper, 12pt]{article}
\usepackage{scrextend, algorithm,algorithmic, amssymb, amsmath, cancel, framed, geometry, listings, lmodern, mathtools,multirow,parskip, paralist,pgfplots,extarrows, ragged2e,setspace,tcolorbox, upquote, tikz, enumitem}
\setlist{nosep}
\usepackage{caption}
\usepackage{subcaption}
\usepackage{newtxmath}
\usepackage[english]{babel}
\usetikzlibrary{positioning}
\usepackage{xcolor}
\definecolor{mypink1}{RGB}{0, 170, 0}
\usepackage[utf8x]{inputenc}
\usepackage{fancyhdr}
\fancyhf{}
\usepackage{tabu}
\pagestyle{fancy}
\rfoot{Page \thepage}
\usepackage[T1]{fontenc}
\usepackage[colorlinks]{hyperref}
\hypersetup{
	colorlinks,
	filecolor=magenta,
	linkcolor=blue,
	citecolor=mypink1,      
	urlcolor=cyan,
}
\geometry{
	a4paper,
	total={170mm,257mm},
	left=20mm,
	top=20mm,
}
\usepackage{caption}
\usepackage{xurl}
%\usepackage{calrsfs}
%\DeclareMathAlphabet{\pazocal}{OMS}{zplm}{m}{n}
\newcommand{\stencilpt}[4][]{\node[circle,draw,inner sep=0.1em,minimum size=0.6cm,font=\tiny,#1] at (#2) (#3) {#4}}
\newcommand{\stencilptbig}[4][]{\node[circle,draw,inner sep=0.1em, outer sep=0pt, minimum size=0.7cm,font=\normalfont,#1] at (#2) (#3) {#4}}
\definecolor{shadecolor}{RGB}{240,248,255}
\definecolor{darkred}{rgb}{0.85,0,0}
\setlength{\parskip}{0.5em}
\begin{document}
	\setlength{\parindent}{0cm}
	\renewcommand{\baselinestretch}{0.5}
	\title{Assignment 01}
	\author{Basil Peterhans, Eric Buffle, Linard Büchler}
	\date{\today}
	\maketitle
	\tableofcontents{}
\section{Ray-plane intersection}
A ray equation can be formulated as follows:
$$
\textbf{r}(t) = \textbf{o}+ t\textbf{d}
$$
For a plane $\textbf{n}$, the ray-plane intersection can be formulated as:
$$
\textbf{n}^\top (\textbf{o}+ t\textbf{d})- b = 0
$$
solving for $t$, we get
\begin{align*}
\textbf{n}^\top (\textbf{o}+ t\textbf{d})- b &= 0\\
\textbf{n}^\top\textbf{o} + t\textbf{n}^\top\textbf{d} - b &=0\\
\implies t\textbf{n}^\top\textbf{d} &=b - \textbf{n}^\top\textbf{o}  &=0\\
\implies t &=\frac{b - \textbf{n}^\top\textbf{o}}{\textbf{n}^\top\textbf{d}} \\
\end{align*}
\section{Ray-cylinder intersection + normal derivations}
A cylinder can be described by
\begin{itemize}
  \item A position vector $\textbf{p}_1$ describing the first end point of the long axis of the cylinder
  \item A position vector $\textbf{p}_2$ describing the second end point of the long axis of the cylinder
  \item a radius $r$
\end{itemize}
The axis of the cylinder can be written as $\Delta \textbf{p} = \textbf{p}_1 - \textbf{p}_2$ 
Source : \cite{Pfister_Harvard}

\begin{thebibliography}{10} % 10 is a random guess of the total number of references 
\bibitem{Pfister_Harvard} \url{https://www.doc.ic.ac.uk/~dfg/graphics/graphics2010/GraphicsLecture11.pdf}%Companion}, Addison-Wesley, Reading, MA, 1994.
%
%\bibitem{HK} Kopka, H., Daly P.W., \emph{A Guide to LaTeX},
%Addison-Wesley, Reading, MA, 1999.
%
%\bibitem{PR} Paul Reimer, \emph{http://www.phy.anl.gov/mep/SeaQuest/}, 2014. 
%
%\bibitem{JA} Joseph C. Amato and Roger E. Williams, \emph{AAPT Apparatus Competition}, 2008. 

\end{thebibliography}
\end{document}

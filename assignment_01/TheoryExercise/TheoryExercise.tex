\documentclass[a4paper, 12pt]{article}
\usepackage{scrextend, algorithm,algorithmic, amssymb, amsmath, cancel, framed, geometry, listings, lmodern, mathtools,multirow,parskip, paralist,pgfplots,extarrows, ragged2e,setspace,tcolorbox, upquote, tikz, enumitem}
\setlist{nosep}
\usepackage{caption}
\usepackage{subcaption}
\usepackage{newtxmath}
\usepackage[english]{babel}
\usetikzlibrary{positioning}
\usepackage{xcolor}
\definecolor{mypink1}{RGB}{0, 170, 0}
\usepackage[utf8x]{inputenc}
\usepackage{fancyhdr}
\fancyhf{}
\usepackage{tabu}
\pagestyle{fancy}
\rfoot{Page \thepage}
\usepackage[T1]{fontenc}
\usepackage[colorlinks]{hyperref}
\hypersetup{
	colorlinks,
	filecolor=magenta,
	linkcolor=blue,
	citecolor=mypink1,      
	urlcolor=cyan,
}
\geometry{
	a4paper,
	total={170mm,257mm},
	left=20mm,
	top=20mm,
}
\usepackage{caption}
\usepackage{xurl}
%\usepackage{calrsfs}
%\DeclareMathAlphabet{\pazocal}{OMS}{zplm}{m}{n}
\newcommand{\stencilpt}[4][]{\node[circle,draw,inner sep=0.1em,minimum size=0.6cm,font=\tiny,#1] at (#2) (#3) {#4}}
\newcommand{\stencilptbig}[4][]{\node[circle,draw,inner sep=0.1em, outer sep=0pt, minimum size=0.7cm,font=\normalfont,#1] at (#2) (#3) {#4}}
\definecolor{shadecolor}{RGB}{240,248,255}
\definecolor{darkred}{rgb}{0.85,0,0}
\setlength{\parskip}{0.5em}
\begin{document}
	\setlength{\parindent}{0cm}
	\renewcommand{\baselinestretch}{0.5}
	\title{Assignment 01}
	\author{ Linard Büchler, Eric Buffle, Basil Peterhans}
	\date{\today}
	\maketitle
	\tableofcontents{}
\section{Ray-plane intersection}
A ray equation can be formulated as follows:
$$
\textbf{s}(t) = \textbf{o}+ t\textbf{d}
$$
For a plane $\textbf{n}$, the ray-plane intersection can be formulated as:
$$
\textbf{n}^\top (\textbf{o}+ t\textbf{d})- b = 0
$$
solving for $t$, we get
\begin{align*}
\textbf{n}^\top (\textbf{o}+ t\textbf{d})- b &= 0\\
\textbf{n}^\top\textbf{o} + t\textbf{n}^\top\textbf{d} - b &=0\\
\implies t\textbf{n}^\top\textbf{d} &=b - \textbf{n}^\top\textbf{o}\\
\implies t &=\frac{b - \textbf{n}^\top\textbf{o}}{\textbf{n}^\top\textbf{d}} \\
\end{align*}
\section{Ray-cylinder intersection}
The implicit equation for a cylinder centered on $z$ with radius $r$ is
$$
x^2 + y^2 = r^2
$$
where
\begin{align*}
  x&=r\cos\phi\\
  y&=r\sin\phi
\end{align*}
Now we can write
$$
\|\textbf{q} - \textbf{p}_a - (\textbf{v}_a \cdot (\textbf{q} - \textbf{p}_a))\textbf{v}_a\|^2 - r^2 = 0
$$
\begin{itemize}
  \item $r$: radius of a cylinder 
  \item $\textbf{p}_a$: cylinder center
  \item $\textbf{v}_a$: cylinder axis
  \item $\textbf{q}$: point on the cylinder
\end{itemize}
to find intersection points with a ray $\textbf{p} + t\textbf{v}$, substitute $\textbf{q} = \textbf{p} + t\textbf{v}$ 
\begin{align*}
\|\textbf{p} + t\textbf{v} - \textbf{p}_a - (\textbf{v}_a \cdot (\textbf{p} + t\textbf{v} - \textbf{p}_a))\textbf{v}_a\|^2 - r^2 &= 0\\
\|\textbf{p} + t\textbf{v} - \textbf{p}_a - (\textbf{v}_a \cdot (\textbf{p}  -\textbf{p}_a) + \textbf{v}_a\cdot t\textbf{v} ))\textbf{v}_a\|^2 - r^2 &= 0\\
\|\textbf{p} + t\textbf{v} - \textbf{p}_a - (\textbf{v}_a \cdot (\textbf{p}  -\textbf{p}_a))\textbf{v}_a - (\textbf{v}_a\cdot t\textbf{v} ))\textbf{v}_a\|^2 - r^2 &= 0\\
\|t\underbrace{(\textbf{v} - (\textbf{v}_a\cdot \textbf{v} ))\textbf{v}_a)}_{\textbf{F}} + \underbrace{\textbf{p}  - \textbf{p}_a - (\textbf{v}_a \cdot (\textbf{p}  -\textbf{p}_a))\textbf{v}_a}_{\textbf{G}} \|^2 - r^2 &= 0\\
\|t\textbf{F} +\textbf{G} \|^2 - r^2 &= 0\\
t^2\underbrace{\textbf{F}^\top\textbf{F}}_{a} +t\underbrace{2\textbf{F}^\top\textbf{G}}_{b} + \underbrace{\textbf{H}^\top\textbf{H}  - r^2}_{c} &= 0
\end{align*}
This is a quadratic equation with the following coefficients:
\begin{itemize}
  \item $a = \|\textbf{v} - (\textbf{v}_a\cdot \textbf{v} ))\textbf{v}_a\|^2$
\item $b = 2\Big(\textbf{v} - (\textbf{v}_a\cdot \textbf{v} ))\textbf{v}_a\Big)^\top \Big(\textbf{p}  - \textbf{p}_a - (\textbf{v}_a \cdot (\textbf{p}  -\textbf{p}_a))\textbf{v}_a\Big)$
\item  $ c = \|\textbf{p}  - \textbf{p}_a - (\textbf{v}_a \cdot (\textbf{p}  -\textbf{p}_a))\textbf{v}_a\|^2 - r^2$
\end{itemize}

\section{Intersection Normal Vector}
\begin{itemize}
\item $\textbf{c2}$: projection of intersection point to cylinder axis
\item $\textbf{n}$: normal vector on intersection
\end{itemize}	

\begin{align*}
\textbf{c2} =  \textbf{p}_a + (\textbf{q}- \textbf{p}_a \cdot \textbf{v}_a) \textbf{v}_a
\\
\textbf{n} = \frac{\textbf{c2}-\textbf{q}}{ r} 
\end{align*}






\begin{thebibliography}{10} % 10 is a random guess of the total number of references 
  \bibitem{Pharr_1999} Denis Zorin, \url{https://mrl.nyu.edu/~dzorin/rend05/lecture2.pdf}%Companion}, Addison-Wesley, Reading, MA, 1994.

\end{thebibliography}
\end{document}
